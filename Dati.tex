%----------------------------------------------------------------------------------------
%   USEFUL COMMANDS
%----------------------------------------------------------------------------------------

\newcommand{\dipartimento}{Dipartimento di Matematica ``Tullio Levi-Civita''}

%----------------------------------------------------------------------------------------
% 	USER DATA
%----------------------------------------------------------------------------------------

% Data di approvazione del piano da parte del tutor interno
% compilare inserendo al posto di GG 2 cifre per il giorno, e al posto di 
% AAAA 4 cifre per l'anno
\newcommand{\dataApprovazione}{GG Mese AAAA}

\newcommand{\nomeStudente}{}
\newcommand{\cognomeStudente}{}
\newcommand{\matricolaStudente}{}
\newcommand{\emailStudente}{}
\newcommand{\telStudente}{}

\newcommand{\nomeTutorAziendale}{}
\newcommand{\cognomeTutorAziendale}{}
\newcommand{\emailTutorAziendale}{}
\newcommand{\telTutorAziendale}{}
\newcommand{\ruoloTutorAziendale}{}

\newcommand{\ragioneSocAzienda}{Azienda S.p.A}
\newcommand{\indirizzoAzienda}{Via Roma 1, Roma (RM)}
\newcommand{\sitoAzienda}{http://example.com/}

\newcommand{\titoloTutorInterno}{Prof.}
\newcommand{\nomeTutorInterno}{NomeDocente}
\newcommand{\cognomeTutorInterno}{CognomeDocente}


\newcommand{\scopoStage}{
    % Personalizzare inserendo lo scopo dello stage, cioè una breve descrizione
Lo scopo di questo progetto di stage è ... .\\
Lo studente avrà il compito di ... .\\
}

\newcommand{\interazioneStudenteTutor}{
    % Personalizzare definendo le modalità di interazione col tutor aziendale
Regolarmente, (almeno una volta la settimana) ci saranno incontri diretti con il tutor aziendale \nomeTutorAziendale\ 
\cognomeTutorAziendale\ e stakeholders per verificare lo stato di avanzamento, chiarire eventualmente  gli obiettivi, affinare la ricerca e aggiornare il piano stesso di lavoro.
}

\newcommand{\prodottiAttesi}{
    % Personalizzare definendo i prodotti attesi (facoltativo)
Lo studente dovrà produrre una relazione scritta che illustri i seguenti punti.
\begin{enumerate}
	\item Primo punto \\
		  Descrizione. 

	\item Secondo punto \\
		  Descrizione.

	\item Terzo punto.
		  Descrizione.
\end{enumerate}

Nel qual caso in cui lo studente, in seguito all'analisi, abbia ancora tempo a sua disposizione ... .
}

\newcommand{\contenutiFormativi}{
    % Personalizzare indicando le tecnologie e gli ambiti di interesse dello stage
Durante questo progetto di stage lo studente avrà occasione di approfondire le sue conoscenze nell'ambito ... .
}

\newcommand{\prospettoSettimanale}{
     % Personalizzare indicando in lista, i vari task settimana per settimana
    \begin{itemize}
        \item \textbf{Prima Settimana}
        \begin{itemize}
            \item Incontro con persone coinvolte nel progetto per discutere i requisiti e le richieste
            relativamente al sistema da sviluppare;
            \item Verifica credenziali e strumenti di lavoro assegnati;
            \item Presa visione dell’infrastruttura esistente;
            \item Formazione sulle tecnologie adottate;
        \end{itemize}
        \item \textbf{Seconda Settimana - Sottotitolo} 
        \begin{itemize}
            \item ;
        \end{itemize}
        \item \textbf{Terza Settimana - Sottotitolo} 
        \begin{itemize}
            \item ;
        \end{itemize}
        \item \textbf{Quarta Settimana - Sottotitolo} 
        \begin{itemize}
            \item ;
        \end{itemize}
        \item \textbf{Quinta Settimana - Sottotitolo} 
        \begin{itemize}
            \item ;
        \end{itemize}
        %\newpage
        \item \textbf{Sesta Settimana - Sottotitolo} 
        \begin{itemize}
            \item ;
        \end{itemize}
        \item \textbf{Settima Settimana - Sottotitolo} 
        \begin{itemize}
            \item ;
        \end{itemize}
        \item \textbf{Ottava Settimana - Conclusione} 
        \begin{itemize}
            \item ;
        \end{itemize}
    \end{itemize}
}

\newcommand{\attivita}{
    % Personalizzare indicando in tabella la ripartizione oraria delle varie attività
    % indicate nel prospettoSettimanale
	 48 & Prima attività \\ \hdashline 
	 \multirow{4}{0cm}\\ 
	 \textit{16} & 
	 \textit{Primo compito} \\
	 \textit{8} & 
	 \textit{Secondo compito} \\ 
	 \textit{8} & 
	 \textit{Terzo compito} \\
	 \textit{16} & 
	 \textit{Quarto compito} \\ 	 
	 \hline

	 40 & Seconda attività \\ \hdashline 
	 \multirow{3}{0cm}\\ 
	 \textit{24} & 
	 \textit{Primo compito} \\
	 \textit{8} & 
	 \textit{Secondo compito} \\
	 \textit{8} & 
	 \textit{Terzo compito} \\
	 \hline
	 
	 80 & Terza attività \\ \hdashline 
	 \multirow{1}{0cm}\\ 
	 \textit{80} & 
	 \textit{Primo compito} \\
	 \hline
	 
	 52 & Quarta attività \\ \hdashline 
	 \multirow{1}{0cm}\\ 
	 \textit{52} & 
	 \textit{Secondo compito} \\
	 \hline
	 
	 80 & Quinta attività  \\ \hdashline 
	 \multirow{1}{0cm}\\ 
	 \textit{8} & 
	 \textit{Primo compito} \\
	 \hline
}

% Indicare il totale complessivo (deve essere compreso tra le 300 e le 320 ore)
\newcommand{\totaleOre}{}

\newcommand{\obiettiviObbligatori}{
	 \item \underline{\textit{O01}}: primo obiettivo;
	 \item \underline{\textit{O02}}: secondo obiettivo;
	 \item \underline{\textit{O03}}: terzo obiettivo;
	 
}

\newcommand{\obiettiviDesiderabili}{
	 \item \underline{\textit{D01}}: primo obiettivo;
	 \item \underline{\textit{D02}}: secondo obiettivo;
}

\newcommand{\obiettiviFacoltativi}{
	 \item \underline{\textit{F01}}: primo obiettivo;
	 \item \underline{\textit{F02}}: secondo obiettivo;
	 \item \underline{\textit{F03}}: terzo obiettivo;
}

\newcommand{\pathDiagramma}{img/gantt.png}